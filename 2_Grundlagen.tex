% 2_Grundlagen.tex
\section{Grundlagen}
\label{sec:Grundlagen}

\subsection{Prototyping}
\citeauthor{Floyd.1984} (\citeyear[vgl.][S. 2f.]{Floyd.1984}) definiert Prototyping als eine Methode der Software Entwicklung, die einen Lernprozess darstellt, welcher die Qualität eines Softwaresystems verbessern soll.
Das Experiment von \citeauthor{Boehm.1984} (\citeyear[vgl.][S. 22]{Boehm.1984}), welches einen Protyping Ansatz gegenüber reinen Spezifikationen untersucht, unterstreicht dies.
\citeauthor{Boehm.1984} stellt aber auch die Defizite eines reinen Prototyping Ansatzes dar.
Der Prozess besteht aus den Schritten: funktionale Auswahl, Konstruktion, Evaluierung und Weiterverwendung \autocite[vgl.][S. 4]{Floyd.1984}.

Dabei kann der Fokus beim Prototyping auf verschiedenen Aspekten liegen.
Einerseits kann dieser auf der User Experience der Benutzerschnittstelle eines Softwaresystems liegen - indem dient der Prototyp als Kommunikationsmedium zwischen den Nutzern und Entwicklern.
Andererseits kann der Fokus auf der technischen Umsetzbarkeit liegen. Beispielsweise können bestimmte Eigenschaften oder die generelle Machbarkeit eines Softwaresystems untersucht werden \autocite[vgl.][S. 3]{Floyd.1984} \autocite[vgl.][S. 90ff.]{Budde.1990}.

Prototypen können nach dem funktionalen Umfang unterschieden werden.
Ein vertikaler Prototyp ist eine komplette Implementierung eines begrenzten Ausschnitts der Funktionalität \autocite[vgl.][S. 4]{Floyd.1984} \autocite[vgl.][S. 94]{Budde.1990}.
Ein horizontaler Prototyp ist nur eine partielle Implementierung der gesamten Funktionalität. Beispielsweise nur eine Schicht eines Systems \autocite[vgl.][S. 4]{Floyd.1984} \autocite[vgl.][S. 94]{Budde.1990}.

Prototypen lassen sich auch nach ihrem Ziel bzw. Ansatz unterscheiden.
Es gibt bei dieser Klassifizierung drei Arten:
\begin{itemize}
	\item explorativ - Prototyping wird genutzt um unklare Anforderungen zu spezifizieren,
	\item experimentell - Prototyping wird genutzt um die technische Implementierung zu untersuchen und
	\item evolutionär - Prototyping wird iterativ angewendet, indem der Prototyp über mehrere Zyklen angepasst wird  \autocite[vgl.][S. 6]{Floyd.1984} \autocite[vgl.][S. 93]{Budde.1990}.
\end{itemize}

Zur Evaluation von Prototypen nennt \citeauthor{Spitta.1989} in \autocite{Spitta.1989} vier Prüfmethoden:
\begin{itemize}
	\item Test - Erfüllung definierter Anforderungen,
	\item Inspektion - Ansicht der Prototypen mit schriftlicher Dokumentation ohne Diskussion,
	\item Review - Ansicht der Prototypen von mehreren Personen mit Diskussion
	\item Mitarbeit - Die Einschätzung aus der Mitarbeit von Mitgliedern aus dem Entwicklungsteam \autocite[vgl.][S. 21]{Spitta.1989}.
\end{itemize}

Die Literatur in diesem Kapitel ist bei Betrachtung des Veröffentlichungsdatum nicht aktuell.
Dennoch leistet die Verwendete Literatur fundamentale Definitionen für Prototyping.
Zudem nennt \citeauthor{Spitta.1989} die Konferenz \autocite{Budde.1984} als Ansatzpunkt, um einen ''guten Überblick'' \autocite[S. 5]{Spitta.1989} zu bekommen.

Als Methode zur Entscheidungsfindung in der Software Entwicklung und Prototyping erläutert \citeauthor{Spitta.1989} die Nutzwertanalyse (\citeyear[vgl.][S. 89]{Spitta.1989}), welche im nächsten Unterabschnitt vorgestellt wird.

\subsection{Nutzwertanalyse}
Die Nutzwertanalyse ist eine Methode zur Entscheidungsfindung unter mehreren Entscheidungsalternativen \autocite[vgl.][S. 45]{Zangemeister.1976} \autocite[vgl.][S. 74]{Kuhnapfel.2017}.
Dabei dient die Entscheidungsfindung dazu, ein bestimmtes Ziel zu erreichen und nicht die Entscheidung selbst durchzuführen.
Ein Ziel mit seinen Nebenbedingungen dient dazu nicht den Eigentlichen Zweck der Entscheidung zu verfehlen \autocite[vgl.][S. 75]{Kuhnapfel.2017}.

Bei der Nutzwertanalyse können sowohl quantitative als auch nicht-quantitative Aspekte in die Entscheidungsfindung einfließen \autocite[vgl.][S. 74]{Kuhnapfel.2017}.
Die Entscheidung wird dafür in unterschiedliche Entscheidungskriterien aufgelöst.
Jedes Entscheidungskriterium jeder Entscheidungsalternative wird mit ein Nutzwert bewertet \autocite[vgl.][S. 77]{Kuhnapfel.2017}.

Der jeweilige Nutzwert liegt innerhalb einer Skala, welche für die jeweilige Nutzwertanalyse festgelegt ist.
Bei nicht quantitativen Aspekten repräsentiert der Nutzwert somit den subjektiven Nutzen eines Entscheidungskriteriums einer Entscheidungsalternative.
Die Entscheidung soll auf diese Weise versachlicht werden.
Eine typisch verwendete Skala ist die Dreipunktskala \autocite[vgl.][S. 83ff.]{Kuhnapfel.2017}.
Sie ist eine Ordinalskala, welche folgende Werte definiert:
\begin{itemize}
	\item 1 - Kriterium ist nicht erfüllt, 
	\item 2 - Kriterium ist mit Einschränkung erfüllt und
	\item 3 - Kriterium ist zufriedenstellend erfüllt \autocite[vgl.][S. 86]{Kuhnapfel.2017}.
\end{itemize}

Die Entscheidungskriterien werden mit einem Gewicht versehen, welches die Geltung an der Entscheidung repräsentieren \autocite[vgl.][S. 78f.]{Kuhnapfel.2017}.
Zur Festlegung der Gewichtung eignet sich die Paarvergleichsmethode.
Es werden hierbei alle Kriterien anhand einer Kreuztabelle miteinander verglichen und das im jeweiligen Paar wichtigere Kriterium notiert.
Aufgrund der Häufigkeit jedes Kriteriums im Paarvergleich, kann ein Gewicht abgeleitet werden \autocite[vgl.][S. 81f.]{Kuhnapfel.2017}.

Die Entscheidung wird Anhand der summierten gewichteten Nutzwerte für die Entscheidungsalternativen determiniert.
Dabei gilt, dass die Entscheidungsalternative mit der größten Summe der gewichteten Nutzwerte, die am günstigste Entscheidungsalternative der Entscheidung ist \autocite[vgl.][S. 87f.]{Kuhnapfel.2017}.

%Optional kann eine Sensibilitätsanalyse durchgeführt werden.
%Bei dieser wird die Robustheit der Ergebnisse validiert, wenn sich Kriteriengewichte oder Bewertungen ändern \autocite[vgl.][S. 88ff.]{Kuhnapfel.2017}.
%
%Nach \citeauthor{Kuhnapfel.2017} (\citeyear{Kuhnapfel.2017}) lässt sich die Nutzwertanalyse als Ablauf von folgenden Schritten darstellen:
%\begin{enumerate}
%	\item ''Festlegung des Zielsystems und etwaiger Nebenbedingungen'' \autocite[S. 75]{Kuhnapfel.2017}
%	\item ''Auswahl der Entscheidungsalternativen'' \autocite[S. 76]{Kuhnapfel.2017}
%	\item ''Bestimmung von Kriterien'' \autocite[S. 77]{Kuhnapfel.2017}
%	\item ''Gewichtung der Kriterien'' \autocite[S. 78]{Kuhnapfel.2017}
%	\item ''Wahl einer Bewertungsskala'' \autocite[S. 83]{Kuhnapfel.2017}
%	\item ''Bewertung der Kriterien'' \autocite[S. 87]{Kuhnapfel.2017}
%	\item ''Mathematische Aggregation der Nutzwerte'' \autocite[S. 87]{Kuhnapfel.2017}
%	\item optional ''Sensibilitätsanalyse'' \autocite[S. 88]{Kuhnapfel.2017}
%\end{enumerate}