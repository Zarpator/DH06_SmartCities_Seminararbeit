\section{Smart Governance}

%Wie ist der Stand bei Smart Governance in Hamburg?

%Welche Beispielprojekte belegen dass?

%Auf welchem Weg ist Hamburg auf dem Weg zur Smart City im Bereich der Smart Governance?


- Aspekte Smart Governance (laut Smart City Wheel):
  - Stadtinfrastruktur
  - digitaler Servide und kooperative Regierung
  - offene und verknüpfte Daten
  - Smarte Regierung
- Open Government
  - Open Government sollte seine Daten (Gov Data) offen legen (Open Data)
- Interaktion zwischen Bevölkerung und Regierung soll durch "Open Government Data" befördert werden
- Definition "Offene Daten"
  - Verfügbarkeit (in zweckmäßiger Form und zu möglichst niedrigen Kosten)
  - Wiederverwendung (daten müssen (auch maschinell) verarbeitbar sein)
  - Universelle Beteiligung (Nutzung jeder Person ermöglicht)
-  Ziele, um Open Government zu erreichen
  - Transparenz
  - Beteiligung
  - Zusammenarbeit
 
 \cite{Fuetterer.2020}