\section{SWOT Analyse}

Der Überblick über die Dimensionen auf dem Weg zur Smart City hat einen Überblick geliefert. Eine Einschätzung der Stärken und Schwächen soll nun gegeben werden und daraus folgend Chancen und Risiken abgeleitet werden.
\subsection{Stärken}

Die größte Stärke der hamburger Entwicklungen ist die \textbf{hohe Anzahl der Projekte}. In diesem Bereich sind keine Städte in Deutschland ähnlich fortschrittlich. Dadurch sind Smart-City-Projekte im Alltag der Bürger allgegenwärtig und werden mehr zur Normalität, als wenn nur vereinzelt Projekte durchgeführt werden. Das hat zur Folge, dass eine \textbf{hohe Akzeptanz }gegenüber diesen und gegenüber neuen Projekten vorherrscht. Bürger nehmen neu entstehende Projekte so schneller auf und integrieren sie eher in ihren Alltag.

\subsection{Schwächen}

Die hohe Anzahl an Projekten wird in Hamburg jedoch noch nicht optimal ausgenutzt. Viele Projekte integrieren keine anderen Projekte der hamburger Smart-City-Initiative. Dadurch werden \textbf{Synergien nicht genutzt}, die die Effektivität der Projekte weiter steigern könnten. Zusätzlich werden teilweise \textbf{nicht alle Stakeholder mit eingebunden}, die für den Erfolg des Projektes nützlich sein könnten. Eine weitere Schwäche ist der Fokus der Projekte. Manche Projekte fokussieren sich nur auf kurzfristige Problemlösungen ohne die Ursache der Probleme zu ermitteln und zu bekämpfen.

\subsection{Chancen}



\subsection{Risiken}
% - Hauptsächlich aus SWOT-Analyse der Präsentation herausziehen
% - Beispiele einbinden

%Mobility
%- viele Projekte
%- manche greifen ineinander, manche nicht
%  - Hafen nicht, Auto Zug etc gut
%  - Ziel ist der Weltkongress für International Transfer Systems 2021
%- ICTS Weltkongress ist maßgeblich für Mobility-Entwicklungen in Hamburg
%- Beispiel Park'n'joy: Hamburg wird für innovative Projekttests ausgewählt
%  - Weil Akzeptanz in Bevölkerung hoch ist
%- Mobility-Projekte lösen eher langfristige Probleme
%- 