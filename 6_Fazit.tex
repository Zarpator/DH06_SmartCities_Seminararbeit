\section{Fazit und Ausblick}
\label{sec:fazit}

Die SWOT-Analyse hat die Stärken, Schwächen, Chancen und Risiken bei der Umsetzung einer Smart City in Hamburg aufgezeigt. Im Fazit soll nun die Fragestellung aus der Motivation beantwortet werden: Ist Hamburg auf einem guten Weg zur Smart City? 

Wie aus der SWOT-Analyse hervorgeht sticht Hamburg vor allem mit einer großen Anzahl von Projekten hervor. Ein weiterer Aspekt der für die Fortschrittlichkeit Hamburgs spricht, ist der Smart City Index des \textit{bitkom} (Bundesverband Informationswirtschaft, Telekommunikation und neue Medien). In diesem Ranking belegt Hamburg mit Abstand den ersten Platz (ca. 10 Punkte Vorsprung), mit einer Bewertung von 79,5. Vor allem die Bereiche Mobilität und Gesellschaft stechen mit einem Rating von ca. 90 hervor. \autocite[vgl.][]{bitkom.2019}

Wichtig ist, dass dieser Vorsprung genutzt wird und man die aktuellen Probleme und Risiken beseitigt. Vor allem die Silobildung ist ein erhöhtes Risiko, da ständig neue Projekte veröffentlicht werden, aber diese keinerlei Verknüpfung untereinander anstreben. Zudem sind viele Projekte nur kurzfristig gedacht und lösen kein längerfristiges Problem. Hier sollte man sich überlegen, ob man nicht über den Tellerrand schauen sollte, um die Innovation voranzutreiben.

Um einen Ausblick zu geben, kann man sagen, dass die Stadt Hamburg mit der hohen Anzahl an Projekten und einer erhöhten Akzeptanz der Bevölkerung für neue Innovationen im Stadtbild schon ein gutes Fundament stehen hat. Ziel sollte es in Zukunft sein, neben weiteren Projekten, bestehende Projekte miteinander zu verknüpfen, um Synergien zu nutzen. Mit Blick auf den ITS Weltkongress 2021, kann man sagen, dass Hamburg in den nächsten Jahren auf jeden Fall weiterhin an neuen Technologien arbeiten wird, nicht nur im Bereich der Mobilität. Schließlich soll auch der Rest des Stadtbilds ansprechend und innovativ gestaltet sein.

Hamburg ist demnach auf einem sehr guten Weg zur Smart City und ein Vorreiter in Deutschland. Wenn die aktuellen Risiken, die besprochen wurden, in Zukunft behoben werden, kann das zur Zeit schon sehr hohe Potenzial der Stadt noch besser genutzt werden. Hamburg könnte dann auch weltweit als Paradebeispiel für die Umgestaltung einer klassischen Stadt in eine Smart City stehen.



