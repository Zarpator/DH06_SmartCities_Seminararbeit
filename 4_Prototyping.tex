\section{Prototyping}
\label{sec:Prototyping}

\subsection{Vorgehen und Use-Case}
Für jede einzelne Komponente der Drei-Schichten-Architektur und jeweilige Technologie wird ein Prototyp erstellt, der die Anforderungen an die Komponente umsetzt.
Zusätzlich werden Bibliotheken zur Kartenvisualisierung evaluiert, welche eine Subkomponente der Frontend-Komponente darstellen.
Einerseits wird durch die Implementierung spezielle technische Anforderung zu Geo-Daten Verarbeitung validiert.
Andererseits soll durch die Implementierung Erfahrung durch Mitarbeit gesammelt werden.
Es sollen so mögliche Risiken, (technische) Abhängigkeiten oder Besonderheiten aufgedeckt werden.
Die User-Story, welche die gesamten Anforderungen an die Prototypen definiert ist wie folgt:

\begin{quote}
	Als nicht-angemeldeter Benutzer ist mein Einstieg in das Web-Portal eine Übersichtsseite.
	In der Übersicht kann ich alle Bäume einer Streuobstwiese sehen. 
	Wenn ich einen Baum auswähle, sehe ich eine Detailansicht zu dem Baum. 
	In der Detailansicht kann ich die Pflegehistorie einsehen, sowie die Art des Baumes und die Einträge am schwarzen Brett zu diesem Baum. 
\end{quote}

\subsection{Datenbank}
Für die Datenbank Komponente werden PostgreSQL mit der PostGIS Erweiterung und MariaDB evaluiert, da es sich hierbei um kostenlose Open Source Produkte handelt, welche Geo-Datenverarbeiung unterstützen.
Anforderungen aus der User-Story an diese Komponente sind es einzelne Punkte als Geo-Datentyp zu speichern, welche Bäume repräsentieren.
Darüber hinaus müssen auch Geo-Daten in Form von Polygonen als Geo-Datentyp gespeichert werden, welche Obststreuwiesen bzw. Grundstücke repräsentieren.

Die Implementierung des \textbf{MariaDB} Prototypen erfolgte innerhalb 1 Stunde ohne technische Probleme.
Es sind Geo-Datentypen direkt integriert und die Dokumentation bietet eine gute Unterstützung bei der Implementierung.
Negativ aufgefallen ist, dass \ac{DDL} SQL-Anweisungen implizit einen Commit einer Transaktion durchführen \autocite[vgl.][]{MariaDB.od}.
Dies kann zu ungewollten Verhalten bei Abbrüchen von Schema-Änderungen bei z.B. Aktualisierungen führen.

Die Implementierung des \textbf{PostgreSQL} Prototypen erfolgte innerhalb 1 Stunde 15 Minuten ohne Probleme.
Der geringe Zeitunterschied erklärt sich durch die Installation der PostGIS Erweiterung.
Diese kann entweder selbst kompiliert werden oder aus Paketquellen installiert werden.
Die Dokumentation enthält nicht alle Details, die zur Implementierung notwendig sind.
An der Stelle ist die Online-Community\footnote{z.B. Stack Overflow} sehr hilfreich.
Somit kann dieser Punkt vernachlässigt werden.

\subsection{Server}
Für die Anwendungsserver-Komponente werden Flask, Django, Spring Boot und ExpressJS evaluiert, da es mit allen Frameworks möglich ist Geo-Daten zu verarbeiten bzw. die jeweiligen \ac{ORM} direkt Geo-Datendatentypen der verwendeten Datenbank nutzen können.
Anforderungen aus der User Story an diese Komponente sind einerseits der Zugriff auf die Geo-Daten der Datenbank. Andererseits müssen diese Daten dann über eine HTTP / REST Schnittstelle abrufbar sein.

Die Implementierung der Server Komponente mit \textbf{Django} verlief ohne Schwierigkeiten, da das Framework eine gute Dokumentation hat.
Die verwendete Programmiersprache ist Python.
Die vorgegebene Struktur hilft Zuständigkeiten aufzuteilen.
Die Implementierung hat 3 Stunden gedauert.
Um die Geo-Daten Verarbeitung in Django zu nutzen, müssen noch zusätzliche Pakete in dem Betriebssystem installiert werden.
Django bietet zusätzlich noch die Möglichkeit Administrationsseiten und Benutzerverwaltungen einfach aufzubauen und zu verwenden.

Die Implementierung mit \textbf{Flask} verlief ohne Schwierigkeiten, da die Dokumentation und Online-Community relativ stark sind.
Die Implementierung hat 2 Stunden 30 Minuten gedauert.
Die verwendetet Programmiersprache ist Python.
Negativ aufgefallen ist, dass viele Bibliotheken eingebunden zusätzlich werden müssen, um die Funktionalität der Geo-Datenverarbeitung auf der Serverseite zu ermöglichen.
Außerdem kam das Problem von zirkulären Abhängigkeiten auf, welches bei dem Versuch der Auslagerung der \ac{ORM}-Entitäten aus dem Haupt-Modul der Anwendung erfolgte.

Die Implementierung des \textbf{Spring Boot} Prototypen war nicht erfolgreich.
Nach 5 Stunden wurde die Implementierung eingestellt.
Die verwendete Programmiersprache ist Java in Verbindung einer OpenJDK Java Version.
Die Prototyp basiert auf Spring Boot Starter-Konfigurationen für REST / Java Persistence API Anwendungen.
Um die Geo-Datenverarbeitungsfunktion von dem \ac{ORM} Framework Hibernate zu Nutzen, muss der jeweilige SQL Dialekt aus dem Hibernate Spatial Paket konfiguriert werden.
Hierbei kam es zu Fehlern.
Die Spring Boot Anwendung konnte sich nicht mehr selbst konfigurieren und es war auch nicht mehr nachvollziehbar, was diesen Fehler auslöste.
In Online-Entwickler-Communities war keine Lösung zu diesem Problem dokumentiert.

Die Implementierung des \textbf{ExpressJS} Prototypen verlief ohne Probleme innerhalb von 2 Stunden.
Die verwendete Programmiersprache ist JavaScript, welches in der Node.JS Laufzeitumgebung ausgeführt wird.
Als \ac{ORM}-Framework wurde Sequelize verwendet.
Mit Sequelize können native Geo-Datentypen der Datenbanken verwendet werden.
Zur Verwendung von Geo-Daten-Funktionen in Datenabfragen können SQL Statements definiert werden, dessen Ergebnisse auf JavaScript Objekte automatisch transferiert werden.
Die Dokumentation von allen benötigten Komponenten ist sehr unterstützend und ausreichend detailliert für die Implementierung.
Bei größeren Implementierungen muss die Verwendung von asynchronen Funktionsaufrufen\footnote{Funktionsaufrufe werden aus dem Hauptthread der Anwendung in andere Threads verschoben. Dies passiert bei zeitintensiven I/O-, Datenbank- oder Netzwerkaufrufen.
Damit kann der Hauptthread der Anwendung weitere Anfragen bearbeiten ohne durch eben solche Operationen blockiert zu sein. Sobald der asynchrone Aufruf beendet ist, kehrt der Kontrollfluss wieder in den Hauptthread zurück.} in dem Software-Design beachtet werden.

\subsection{Kartenvisualisierung}
Mit der Kartenvisualisierung werden den Anwendern des Web-Portals Streuobstwiesen, Grundstücke und Bäume visualisiert.
Daher muss die Kartenvisualisierung wenigstens Punkte, Polygone und Tile-Layer\footnote{Ein Tile-Layer ist die Karte selbst. Durch eine Aufteilung in Tiles muss nicht die gesamte Weltkarte angezeigt werden.} anzeigen.
Es werden die Bibliotheken Leaflet, Modestmaps und Polymaps evaluiert.

Der Prototyp wurde mit der Bibliothek \textbf{Leaflet} implementiert.
Die Implementierung war gut umzusetzen, da die API und die Dokumentation sehr ausdrucksstark und intuitiv sind.
Insgesamt hat die Implementierung 1 Stunde gedauert.
Als Tile-Layer wurde zu Testzwecken \ac{OSM} verwendet.

Bei der Einarbeitung zu den Bibliotheken \textbf{Modestmaps} und \textbf{Polymaps} haben sich starke Unterschiede zu Leaflet ergeben.
Modestmaps und Polymaps können somit ohne Prototyp evaluiert werden.

\subsection{Frontend}
Für die Frontend-Komponente werden die Frameworks OpenUI5 und Angular evaluiert, da diese mit Abstand die meisten Vorkenntnisse im Entwicklungs-Team haben.
AngularJS wird nicht evaluiert, da Angular selbst der Nachfolger ist und zum Großteil auf die gleichen Konzepte wie AngularJS setzt.
Als Kartenvisualisierung wird in den beiden Prototypen Leaflet mit dem \ac{OSM}-Tile-Layer verwendet.
Aus der User Story ergeben sich die Anforderungen, dass den Benutzern Geo-Informationen präsentiert werden müssen.
Außerdem muss es den Benutzern möglich sein zwischen verschiedenen Sichten navigieren zu können.

Die Implementierung des \textbf{OpenUI5} Prototypen erfolgte ohne Probleme innerhalb von 8 Stunden.
Zur Unterstützung von Leaflet musste ein UI-Compoenent entwickelt werden, welches Leaflet einbindet und eine Verbindung von OpenUI5 zu Leaflet ermöglicht.
Zur Entwicklung der Benutzeroberfläche konnte auf viele vorgefertigte Komponenten zurückgegriffen werden, ohne das viele \ac{CSS} Anpassungen gemacht werden müssen.
Die vorgefertigten Benutzeroberflächen-Komponenten und das Layout der Anwendung passen sich dem Endgerätetyp und -größe an.
Es kann außerdem auf das Fiori Design-Konzept für konsistente Oberflächen zurückgegriffen werden, da es von OpenUI5 unterstützt wird.

Die Implementierung des \textbf{Angular} Prototypen erfolgte ohne Probleme innerhalb von 9 Stunden.
Zur Integration von Leaflet kann auf eine Komponente zurückgegriffen werden, welche Leaflet in Angular als Benutzeroberflächen-Komponente verfügbar macht.
Dennoch muss Leaflet in eigenen Initialisierungsevents explizit initialisiert werden.
Außerdem gibt es nicht die Möglichkeit Tile-Layer oder Punkte über Data-Binding zu übergeben.
Zur Entwicklung der Benutzeroberfläche wurde die Material-Design Komponente von Angular verwendet.
Bei der Verwendung der Leaflet-Komponente gab es \ac{CSS} Seiteneffekte, welche durch manuelle \ac{CSS} Anpassungen aufgelöst werden mussten.
Außerdem bietet Angular-Material nicht so viele Layout-Möglichkeiten wie OpenUI5.
Als Alternative bietet sich das \ac{CSS}-Framework Bootstrap an.