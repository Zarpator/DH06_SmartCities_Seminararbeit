\section{Smart Mobility}
\label{sec:smart_mobility}

\subsection{Motivation und Aufbau}

Im Jahr 2021 wird Hamburg den ITS Weltkongress austragen, einen internationalen Kongress für intelligente Transportsysteme.
Ins Leben gerufen wurde er von ERTICO, einer europäischen Organisation, die für die Einführung von Telematik bei Transportsystemen in Europa verantwortlich ist.
Hauptziel des Kongresses soll es sein, Politiker, Experten und die allgemeine Öffentlichkeit auf die Thematik der Smart Mobility aufmerksam zu machen.
Um eine solche Veranstaltung austragen zu können, sollten innovative Ansätze und Ideen vorgestellt werden können.
Ist Hamburg auf dem richtigen Weg, eine solche smarte Mobilität anzubieten? \autocite[vgl.][]{ITSWorldCongress.2020}

Im folgenden Kapitel sollen drei verschiedene Projekte vorgestellt werden, die charakteristisch für die Entwicklung der Smart Mobility in Hamburg sind.
Zwei davon sind spezifisch für den ITS Weltkongress ins Leben gerufen worden.
Im Fazit wird dann kurz erörtert, ob Hamburg in Sachen Smart Mobility überzeugen kann und vielleicht sogar als Vorreiter für andere Städte agiert. 

\subsection{HEAT}

HEAT steht für „Hamburg Electronic Autonomous Transportation” und ist ein Projekt der Hamburger Hochbahn.
Es geht dabei um vollautomatisierte, autonome Kleinbusse, die elektrisch betrieben werden. In der Hafencity von Hamburg soll eine Teststrecke und ein Testbetrieb aufgestellt werden. HEAT zählt zu den Projekten, die für den ITS Weltkongress 2021 ins Leben gerufen wurden. \autocite[vgl.][]{SmartCityKompass.2020c}

In einer vierjährigen Testphase, die von 2017 bis 2021 andauert, sollen etappenweise die Anforderungen an das System erhöht werden.
Dieser Zeitraum ist in drei Phasen unterteilt, bei denen nach und nach die Teststrecke erweitert wird, die Funktionen des autonomen Systems erhöht und die Sicherheitsstufen höher geschraubt werden.
Im Endeffekt soll in diesem Testzeitraum die Technologie genauer erforscht werden und geklärt werden, inwiefern es möglich ist ein solches, autonomes System in den öffentlichen Nahverkehr einzubinden. \autocite[vgl.][]{HOCHBAHN.2020}

Schon 2019 wurden erfolgreiche Tests an HEAT mit Fahrgästen durchgeführt.
Da es weltweit eines der ersten Projekte dieser Art ist, ist HEAT ein absoluter Vorreiter im Bereich des autonomen Personennahverkehrs.
Es zeigt, dass Hamburg fähig ist, sehr innovative und zukunftsträchtige Technologien zu entwickeln und in die bestehende Stadt einzubinden.

\subsection{Park and Joy}

Park and Joy ist ein Projekt der Deutschen Telekom, welches den Parkverkehr in Großstädten deutlich verringern soll. Dieser macht laut Studien 30\% des gesamten Stadtverkehrs aus, ein Wert, der optimiert werden kann. Das Programm wird in verschiedenen deutschen Städten angeboten, wobei das Projekt erstmalig in Hamburg erprobt wurde. Auch Park and Joy ist eines von 30 Vorhaben, welches für den ITS Weltkongress initiiert wurde. \autocite[vgl.][]{SmartCityKompass.2020a}

Zentral ist hierbei eine App, die es dem Nutzer vereinfachen soll, einen Parkplatz in der Stadt zu finden. Hierzu ist angedacht bis zu 11.000 Parkplätze mit Sensorik auszustatten, sodass die Anwendung freie Parkplätze identifizieren und den Fahrer dort hin navigieren kann. Es soll dann zusätzlich möglich sein direkt einen Parkschein in der App zu lösen, sodass der aufwändige Weg zum Parkautomaten eingespart werden kann. \autocite[vgl.][]{ParkandJoy.2020}

Allgemein kann man sagen, dass Park and Joy sehr gut dem erhöhten Parkverkehr in Großstädten entgegenwirkt. Es handelt sich hierbei aber nur um eine sehr kurzfristig gedachte Lösung. Ein weiter gedachtes Ziel könnte sein, Autos gänzlich aus dem Stadtverkehr zu verdrängen und die Mobilität auf platzsparendere Verkehrsmittel umzustellen. Park and Joy ist also ein Beispiel dafür, wie in Hamburg teilweise bei der Umsetzung einer Smart City nicht auf lange Sicht gedacht wird.

\subsection{Port Monitor Hafen}

Beim Port Monitor Hafen handelt es sich um ein Projekt der HPA (Hamburg Port Authority).
Eine Leitstandsoftware in der Nautischen Zentrale soll den gesamten Schiffsverkehr im Hamburger Hafen überwachen. Hauptziel ist es hierbei einen vollständigen Überblick über das aktuelle Verkehrsgeschehen zu Wasser zu geben.
Es werden hierzu diverse Informationen wie beispielsweise Liegeplätze, Brückenhöhen oder auch aktuelle Tauchgänge gesammelt und aufbereitet dargestellt.
Ebenfalls ist die Applikation mobil verfügbar, sodass brandaktuelle Geschehnisse in Echtzeit in der Nautischen Zentrale zur Kenntnis genommen werden können. \autocite[vgl.][]{SmartCityKompass.2020b}

Der Port Monitor Hafen wirkt auf den ersten Blick wie eine handelsübliche Monitoringanwendung.
Im Kontext des Hamburger Hafens, mit einem enormen Verkehrsaufkommen, ist es aber eine hochkomplexe Anwendung, die zu Recht als Innovation beschrieben werden kann.
Andererseits ist die Anwendung nicht mit anderen Projekten im Landverkehr verbunden, was eine Chance in der Zukunft wäre.

\subsection{Fazit}

Die Mobilität in Hamburg ist allgemein gesprochen sehr fortschrittlich. Aufgrund des ITS Weltkongresses 2021, wurden sehr viele innovative Projekte ins Leben gerufen.
Wegen des geringen Umfangs konnten in dieser Arbeit aber nur zwei von insgesamt 30 Vorhaben vorgestellt werden, die aber sehr gut zeigen in welche Richtung sich Hamburg zur Zeit entwickelt. 

Schaut man sich die einzelnen Projekte im Smart City Kompass an, fällt auf, dass vor allem die Projekte, die im Bereich des Personennahverkehrs angesiedelt sind, sehr gut miteinander harmonieren \autocite[vgl.][]{SmartCityKompass.2020d}.
Man kann hier von einer gelungenen Umsetzung von intermodalem Verkehr sprechen. Park and Joy zeigt einen anderen Vorteil von Smart Mobility in Hamburg auf: Aufgrund der hohen Bürgerakzeptant und der bereits fortschrittlichen Infrastruktur siedeln große Unternehmen (Deutsche Telekom) ihre Innovationsprojekte in Hamburg an.

Der Port Monitor Hafen stellt eine sehr sinnvolle Lösung für die Organisation des Hafens dar. Allerdings wird diese Lösung auf den Schiffsverkehr beschränkt, Integration mit anderen Mobilitätsbereichen wurde nicht berücksichtigt. Hier könnte Hamburg an Ideen arbeiten, die das Zusammenspiel zwischen Land- und Wasserverkehr optimieren.
