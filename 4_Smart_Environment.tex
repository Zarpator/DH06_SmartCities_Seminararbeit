\section{Smart Environment}
\label{sec:smart_environment}

\subsection{Motivation und Grundlagen}
% was ist smart environment
Smart Environment ist der Bereich im Smart City Wheel nach Cohen, welches sich mit Aspekten der Stadtplanung beschäftigt.
Diese Bereiche sind:
\begin{itemize}
	\item Smart Buildings - energieeffiziente Häuser,
	\item Smart Grids bzw Green Energy - nachhaltige und dezentrale Energieversorgung und
	\item Green Urban Planning - Stadtplanung mit Fokus auf grüne Flächen in den Städten.
\end{itemize}

% ziele von Smart Environment
Ziel für den Bereich Smart Environment im Smart City Wheel ist es, die Stadtplanung und Stadtentwicklung nachhaltig zu gestalten.
In Folge daraus soll eine Smart City ihren Bürgern eine lebenswerte Umgebung bieten.
Der Fokus liegt dabei  darauf, die jeweilige Stadt der Natur anzunähern.
Das umfasst auch den CO2 Ausstoß bzw. Schadstoffe in der Luft zu reduzieren \autocite[vgl.][S. 4]{Monzon.2015}.
Smart Environment stellt nur eine von mehreren Dimensionen dar, um das Ziel einer Smart City zu erreichen - eine hohe Lebensqualität auf Basis nachhaltiger und robuster Stadtentwicklung \autocite[vgl.][S. 3]{Monzon.2015}.

% umsetzung von Smart Environment
Wie alle Teilbereiche des Smart City Konzeptes liegt der Fokus der Implementierung auf der Verwendung von moderner Informations- und Kommunikationstechnologie.
Das alleine reicht aber nicht, um das maximal Mögliche in nachhaltiger und robuster Stadtentwicklung zu erreichen.
Um dies zu erreichen, werden auch neue Entwicklungen von nachhaltigen Baustoffen bzw. Bautechniken benötigt.
Wie bei dem Smart City Konzept auch, ist Smart Environment nur mit Vernetzung zu erreichen.
Nicht nur auf technischer, sondern auch auf wissenschaftlicher und gesellschaftlicher Ebene.

% schnittpunkte mit anderen bereichen des Smart city wheel
Der interdisziplinäre Charakter des Teilbereichs Smart Environments wird deutlich, wenn Lösungskonzepte in diesem Bereich erarbeitet werden.
Es zeigt sich, dass es ausgeprägte Schnittstellen zu den Teilbereichen Smart Mobility und auch Smart Governance gibt.
Smart Mobility beispielsweise aufgrund der Energienachfrage durch Elektromobilität.
Smart Governance beispielsweise durch Akzeptanz der Bürger, welche für die Verwendung von Plattformen wie dem Transparenzportal notwendig ist.

Auch wenn es nicht explizit im Smart City Wheel erwähnt wird, ist auch die Wasserversorgung ein wichtiger Aspekt, welcher sich dem Teilbereich Smart Environment zuordnen lässt \autocite[vgl.][]{Dickey.2018}.

\subsection{Smart Buildings}
% my smart life projekt
Beispielhaft für die Implementierung von Smart Buildings in der Stadt Hamburg ist das Projekt mySMARTLife.
Hierzu wurde als Teil des Projektes ein Viertel im Stadtteil Hamburg Bergedorf mit neuen energieeffizienten Wohnungen errichtet \autocite[vgl.][S. 2ff.]{Hamburg.OD}.
Insgesamt konnte die Stadt Hamburg durch neue Smart Buildings und Umbauten von bestehenden Wohngebäuden den Energieverbrauch um 38\% senken \autocite[vgl.][]{Hamburg.ODB}.
Die Gebäude, die im Rahmen des mySMARTLife Projektes entstanden sind, setzen auf eine Kombination aus modernen Baumaterialien und Smart Home Geräten zur Optimierung des Energieverbrauchs \autocite[vgl.][S. 2ff.]{Hamburg.OD}.
Herausforderung in Hamburg ist der Umbau von denkmalgeschützten Altbauten \autocite[vgl.][]{Hamburg.ODA}.
Außerdem ist es herausfordernd oder nicht möglich alle Wohngebäude in Hamburg umzubauen, da Hamburg eine Großstadt mit 1,9 Millionen Einwohnern ist und auch dementsprechend viel Wohnraum vorhanden ist.


\subsection{Smart Grids und Green Energy}
Das Unternehmen  Stromnetz Hamburg ist der Betreiber des Mittelspannungsnetzes der Stadt Hamburg.
Ziel von  Stromnetz Hamburg ist es, das Stromnetz zu dezentralisieren und nur nachhaltigen Strom (Windenergie, Wasserkraft, Biomasse) einzuspeisen \autocite[vgl.][]{StromnetzHamburg.OD}.
Im Rahmen von mySMARTLife und der neugebauten Siedlung in Hamburg Bergedorf besitzt jedes Haus Solarpanele.
Der damit erzeugte Strom kann von den Bewohnern des jeweiligen Hauses verwendet werden und überschüssiger Strom wird in das Stromnetz eingespeist \autocite[vgl.][S. 9]{Hamburg.OD}.
Stromnetz Hamburg hat ihr Netz mit Sensoren ausgestattet, um auf die Anforderungen eines dezentralisierten Netzes einzugehen \autocite[vgl.][]{StromnetzHamburg.OD}.

Neben dem Stromnetz wird auch das Fernwärmenetz der Stadt Hamburg dezentralisiert und modernisiert werden.
Das Unternehmen Hamburg Energie sieht in bisherigen Konzepten zur Fernwärme Optimierungspotenzial.
Beispielsweise soll so auch Geothermie, Solartenergie und Biomasse genutzt werden.
Aktuell sollen die Konzepte in dem Stadtteil Hamburg-Wilhelmsburg erprobt werden \autocite[][]{HamburgEnergie.OD}.

Neben der Versorgung von Wohngebäuden wird mit der Landstromanlage für Kreuzfahrtschiffe versucht, Kreuzfahrtschiffe emissionsfrei während ihrer Zeit in Hamburg mit Energie zu versorgen.
Ohne eine Landstromanlage würden Schiffe dadurch weiter mit ihren Motoren Energie produzieren.
Das führt zu zusätzlichen CO2 Emissionen in der Stadt.
Der Energiebedarf eines Kreuzfahrtschiffes entspricht der einer Kleinstadt mit ca. 75.000 Einwohnern \autocite[vgl.][]{SmartCityKompass.OD}.
Die Landstromanlage in Hamburg steht in Kritik, da sie mit 10 Millionen Euro sehr teuer war und Kreuzfahrtschiffe den Landstrom wenig oder gar nicht nutzen, da dieser teurer ist als selbsterzeugte Energie der Kreuzfahrtschiffe.

\subsection{Green Urban Planning}
Hamburg ist die die grünste Stadt Deutschlands unter den Städten über 500.000 Einwohnern \autocite[vgl.][]{BerlinerMorgenpost.2016}.
Diese Position beruht auf Städtebauentscheidungen, welche seit 1919 getroffen werden, um Hamburgs Bürgern eine Lebenswerte Umgebung zu bieten. Ein Beispiel hierfür ist das grüne Netz, welches grüne Adern von den Wäldern aus der Umgebung in das Stadtinnere führt \autocite[vgl.][]{Hamburg.ODC}.
Um diese natürlichen Bereiche zu erhalten und zu verbreitern, werden Projekte zum Bepflanzen der Dächer von Gebäuden in der Stadt gefördert \autocite[vgl.][]{Hamburg.ODD}.
Es wird so versucht trotz Neubauten, welche eigentlich Flächen einnehmen, Grünflächen zu schaffen.
Damit folgt die Stadt Hamburg der Vision des Künstlers Friedensreich Hundertwasser, der Gebäuden einen möglichst großen bezug zur Natur geben wollte und unter Anderem den Einsatz von Pflanzen auf möglichst vielen Gebäudeflächen forderte.

\subsection{Kritische Betrachtung}
Bei Betrachtung der verschiedenen Implementierungen der Smart City Konzepte im Bereich Smart Environment in der Stadt Hamburg stellt sich heraus, dass die Projekte nicht nur einem Bereich zugeordnet werden können.
Beispielsweise müssen sowohl Smart Mobility als auch Smart Environment und Smart Governance gemeinsam betrachtet werden, wenn es um nachhaltige Städteplanung geht, wie es im Projekt mySMARTLife geschehen ist.
Dieses Vorgehen ist auch für die restlichen Dimensionen des Smart City Wheels angebracht.
Dies zeigt, dass die Vernetzung auf allen Ebenen solcher Projekte ein wichtiger Faktor auf dem Weg zur Smart City ist.
Genau hier liegt aber auch ein Problem der vorgestellten Beispiele.
Die verschiedenen Projekte sind unabhängig voneinander und wenig oder gar nicht vernetzt.
Das birgt das Risiko der Silo Bildung.