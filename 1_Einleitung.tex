%1_Einleitung
\section{Einleitung}

\subsection{Motivation und Problemstellung}
Laut dem Smart City Kompass hat Hamburg stand Juni 2020 die meisten Smart City Projekte in Deutschland \autocite[vgl.][]{SmartCityKompass.ODA}.
Außerdem is Hamburg Teilnehmer an dem europäischen Projekt mySMARTLife, welchem Smart City Konzepte implementiert und evaluiert werden sollen \autocite[vgl.][S. 1ff.]{Hamburg.OD}.
Damit bewegt sich Hamburg in die Richtung einer Smart City.

\begin{quote}[Umfrage aus \autocite[S. 126]{Spil.2017}]{Respondent 2}
	\glqq everybody is talking about the smart city concept. However, the ultimate smart city is an utopia and does not exist\grqq
\end{quote}

Wie das Zitat zeigt, wird das Konzept von Smart Cities nicht ohne Kritik aufgenommen.
Es spiegelt auch wieder, dass sich die Stadt Hamburg auch kritisch und reflektiert mit der Thematik auseinandersetzt.
Daher stellt sich die Frage: Ist Hamburg auf einem guten Weg zu einer Smart City?

\subsection{Aufbau der Arbeit, Ziel und Methodik}
Ziel der Arbeit ist es den aktuellen Stand der Implementierung von Smart City Konzepten unter der Vision einer Smart City zu beurteilen.

Die vorliegende Arbeit behandelt dazu drei Themenfelder aus dem Smart City Wheel nach Boyd Cohen.
Diese Themenfelder sind: Smart Mobility in \autoref{sec:smart_mobility}, Smart Governance in \autoref{sec:smart_governance} und Smart Environment in \autoref{sec:smart_environment}.

Darauf folgend wird in \autoref{sec:swot_analyse} der Stand der Implementierung anhand einer SWOT-Analyse bewertet.
Zum Schluss wird in \autoref{sec:fazit} das Ergebnis der Arbeit kritisch betrachtet und ein Ausblick gegeben.

\subsection{Abgrenzung der Arbeit}
Es werden lediglich Smart Mobility, Smart Governance und Smart Environment als Aspekte des Smart City Wheel in der Arbeit betrachtet.
Weitere Aspekte würden den Umfang der Arbeit übersteigen ohne die gewollte Tiefe der Themen zu reduzieren.
Die Autoren präferieren eine qualitative Auseinandersetzung statt einer quantitativen.