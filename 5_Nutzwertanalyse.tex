\section{Nutzwertanalyse}
\label{sec:Nutzertanalyse}

\subsection{Vorgehen und Skala}
Im folgenden werden die Nutzwertanalysen beschrieben, die zur Entscheidungsfindung der benötigten Komponenten der Technologieauswahl dienen.
Bis auf die Nutzwertanalysen in \autoref{sec:kartenvisualisierung} \nameref{sec:kartenvisualisierung} und \autoref{sec:tileservice} \nameref{sec:tileservice} wird jeweils eine individuelle Gewichtung der Kriterien ermittelt.

Die Unterabschnitte in diesem Abschnitt repräsentieren die Komponenten der Drei-Schichten-Architektur.
\autoref{sec:kartenvisualisierung} \nameref{sec:kartenvisualisierung} und \autoref{sec:tileservice} \nameref{sec:tileservice} Stellen Subkomponenten dar, welche in der Frontendkomponente angesiedelt sind.
	
Innerhalb der Unterabschnitte der jeweiligen Nutzwertanalysen sind die Kriterien mit einzelnen Buchstaben abgekürzt.
Zur Bewertung der Kriterien wird die Dreipunkteskala verwendet.
Sie wird verwendet, da sie nur drei Werte umfasst aber dennoch eine Abstufung zulässt.
Zur Verwendung der Ergebnisse der Vorkenntnisabfrage aus \autoref{sec:Vorkenntnisabfrage} werden die jeweiligen Vorkenntnisse auf einen Wert der Dreipunktskala umgesetzt.
Die Umrechnung ist in Anhang \ref{sec:umrechung} erklärt.

\subsection{Datenbank}
Die Kriterien zur Datenbankauswahl sind in \autoref{tab:db-kriterien} dargestellt.
Kriterium A wird anhand der Vorkenntnisabfrage bewertet.
Kriterium B, C und D sind durch Mitarbeit an der Implementierung des Prototypen zu bewerten.
Kriterium E ist Anhand der Dokumentation von QGIS zu beantworten.
Dieser Aspekt hat Wichtigkeit durch die Anwendung von QGIS als Analyseprogramm für Geo-Daten.

\begin{table}[h]
	\caption{Kriterien der Datenbank Auswahl}
	\begin{center}
		\begin{tabular}{ll}
			\toprule
			Nr. & Kriterium             \\ \midrule
			A   & Vorkenntnisse         \\
			B   & Wiederherstellbarkeit \\
			C   & Funktionsumfang       \\
			D   & Implementierung       \\
			E   & QGIS Integration      \\ \bottomrule
		\end{tabular}
	\end{center}
	\label{tab:db-kriterien}
\end{table}

Bei der Gewichtung der Kriterien stellt sich raus, dass Vorkenntnisse für die Datenbank nur einen kleinen Teil ausmachen.
Wichtiger sind dabei die Wiederherstellbarkeit und die Aufwendigkeit der Implementierung eines Datenmodells, welches Geo-Datentypen verwendet.
Unter dieser Sichtweise wird die Gewichtung der Kriterien festgelegt, welche in \autoref{tab:db-gewichtung} dokumentiert sind.

\begin{table}[h]
	\caption{Gewichtung der Kriterien der Datenbankauswahl}
	\begin{center}
		\begin{tabular}{llllllrr}
			\toprule
			Kriterium & A & B & C & D & E & \multicolumn{1}{l}{Summe} & \multicolumn{1}{l}{Gewicht} \\ \midrule
			A         & - & B & A & D & E &                         1 &                    10,00 \% \\
			B         & - & - & B & B & B &                         4 &                    40,00 \% \\
			C         & - & - & - & D & C &                         1 &                    10,00 \% \\
			D         & - & - & - & - & D &                         3 &                    30,00 \% \\
			E         & - & - & - & - & - &                         1 &                    10,00 \% \\
			%			Summe &   &   &   &   &   &                        10 &                   100,00 \% \\
			\bottomrule
		\end{tabular}
	\end{center}
	\label{tab:db-gewichtung}
\end{table}

Kriterium A wurde anhand der Vorkenntnisse bewertet.
Bei Kriterium B bekommt MariaDB einen Wert von 2, wegen impliziten Commits in Transaktionen bei \ac{DDL} Statements.
Kriterium C und D wurden durch Erfahrung bei Implementierung des Prototypen und Arbeit mit den Dokumentationen der beiden Datenbanken festgelegt.
Laut der Dokumentation von QGIS wird MariaDB nicht unterstützt, dafür PostgreSQL mit der PostGIS Erweiterung \autocite[vgl.][]{qgis.od}.
Die Bewertungen sind in \autoref{tab:db-bewertung} dokumentiert.

\begin{table}[h]
	\caption{Bewertung der Kriterien der Datenbanken Auswahl}
	\begin{center}
		\begin{tabular}{lrr}
			\toprule
			Kriterium & \multicolumn{1}{l}{MariaDB} & \multicolumn{1}{l}{PostgreSQL} \\ \midrule
			A         &                           3 &                              2 \\
			B         &                           2 &                              3 \\
			C         &                           2 &                              3 \\
			D         &                           3 &                              3 \\
			E         &                           1 &                              3 \\ \bottomrule
		\end{tabular}
	\end{center}
	\label{tab:db-bewertung}
\end{table}

Die Nutzwerte sind in \autoref{tab:db-nutzwerte} dargestellt.
In dieser Entscheidungssituation ist die Verwendung von PostgreSQL als Datenbank günstiger als MariaDB.

\begin{table}[h]
	\caption{Nutzwerte der Datenbank Auswahl}
	\begin{center}
		\begin{tabular}{rr}
			\toprule
			MariaDB & PostgreSQL \\ \midrule
			2,3 & 2,9 \\ \bottomrule	
		\end{tabular}
	\end{center}
	\label{tab:db-nutzwerte}
\end{table}

\subsection{Server}
Die Kriterien für die Serverauswahl sind in \autoref{tab:server-kriterien} dargestellt.
Kriterium A und B ergeben sich durch die Vorkenntnisse des Entwicklungsteams.
Die weiteren Kriterien werden anhand der Mitarbeit an der Implementierung der Server Prototypen bewertet.

\begin{table}[h]
	\caption{Kriterien der Server Auswahl}
	\begin{center}
		\begin{tabular}{ll}
			\toprule
			Nr. & Kriterien \\ \midrule
			A & Vorkenntnisse Programmiersprachen \\ 
			B & Vorkenntnisse Framework \\ 
			C & Implementierung – Struktur / Nachvollziehbarkeit \\ 
			D & Implementierung – Aufwand \\ 
			E & ORM Geodatenverarbeitung \\ 
			F & Dokumentation \\ 
			G & Umsetzbarkeit \\ \bottomrule
		\end{tabular}
	\end{center}
	\label{tab:server-kriterien}
\end{table}

Bei der Gewichtung ist die Umsetzbarkeit besonders wichtig.
Sie gibt an, ob es mit dem Prototypen möglich war die Anforderungen der User Story für die Server Komponente umzusetzen.
Auch wichtig ist die Unterstützung von Geo-Datenverarbeitung über den \ac{ORM}, welcher jeweils verwendet wird.
Durch einen \ac{ORM} kann der Aufwand bei der Integration von Datenbanken in Server-Code reduziert werden.
Darüber hinaus ist Vorwissen der Programmiersprache wichtiger als die üblichen Kriterien.

\begin{table}[h]
	\caption{Gewichtung der Kriterien der Server Auswahl}
	\begin{center}
		\begin{tabular}{llllllllrrr}
			\toprule
			      & A & B & C & D & E & F & G & \multicolumn{1}{l}{Summe} & \multicolumn{1}{l}{Gewicht} & \multicolumn{1}{l}{korrigiert} \\ \midrule
			A     & - & A & A & A & E & F & G &                         3 &                    14,00 \% &                       14,00 \% \\
			B     & - & - & B & B & E & F & G &                         2 &                    10,00 \% &                       10,00 \% \\
			C     & - & - & - & C & E & C & G &                         2 &                    10,00 \% &                       10,00 \% \\
			D     & - & - & - & - & D & D & G &                         2 &                    10,00 \% &                       10,00 \% \\
			E     & - & - & - & - & - & E & G &                         4 &                    19,00 \% &                       19,00 \% \\
			F     & - & - & - & - & - & - & G &                         2 &                    10,00 \% &                       10,00 \% \\
			G     & - & - & - & - & - & - & - &                         6 &                    28,57 \% &                       27,00 \% \\
			Summe &   &   &   &   &   &   &   &                        21 &                   101,57 \% &                      100,00 \% \\ \bottomrule
		\end{tabular}
	\end{center}
	\label{tab:server-gewichtung}
\end{table}

In \autoref{tab:server-bewertung} ist die Bewertung der Kriterien.
Bei Kriterium C ist ExpressJS mit 3 zu bewerten, da ExpressJS prägnante und intuitive Schnittstellen haben. Django ist in diesem Kriterium auch mit 3 zu bewerten, da Django eine gewisse Struktur der Module vorgibt und so hilft den Code zu strukturieren.
Flask wird mit 2 bewertet, da sich bei Auslagerung von Modulen aus dem Hauptmodul Potential zu zyklischen Abhängigkeiten ergibt.

Bei Kriterium D werden ExpressJS und Flask mit 3 bewertet, da der Aufwand überschaubar ist.
Django ist mit 2 zu bewerten, da es noch zusätzliche Konfigurationsschritte notwendig sind, um eine Anwendung initial aufzusetzen. Dies hat sich auch im Zeitaufwand der Prototypen Implementierung manifestiert.
Spring Boot ist mit 1 zu bewerten, da nach 5 Stunden Implementierung noch kein Ergebnis vorliegt.

Bei Kriterium E sind bis auf Spring Boot alle Alternativen mit 3 zu bewerten, da alle verwendeten \ac{ORM} Geo-Datenverarbeitung bzw. Geo-Datentypen unterstützen.
Bei Spring Boot mit Hibernate Spatial ist dies theoretisch möglich, aber Aufgrund des Problems bei der Implementierung ist dieser Aspekt mit 2 bewertet.

\begin{table}[h]
	\caption{Bewertung der Kriterien der Server Auswahl}
	\begin{center}
		\begin{tabular}{lrrrr}
			\toprule
			Kriterium & \multicolumn{1}{l}{ExpressJS} & \multicolumn{1}{l}{SpringBoot} & \multicolumn{1}{l}{Django} & \multicolumn{1}{l}{Flask} \\ \midrule
			A         &                             3 &                              3 &                          2 &                         2 \\
			B         &                             2 &                              3 &                          2 &                         2 \\
			C         &                             3 &                              2 &                          3 &                         2 \\
			D         &                             3 &                              1 &                          2 &                         3 \\
			E         &                             3 &                              2 &                          3 &                         3 \\
			F         &                             2 &                              2 &                          3 &                         3 \\
			G         &                             3 &                              1 &                          3 &                         3 \\ \bottomrule
		\end{tabular}
	\end{center}
	\label{tab:server-bewertung}
\end{table}

In \autoref{tab:server-nutzwerte} sind die Nutzwerte der Entscheidungsalternativen.
ExpressJS stellt sich als günstigste Alternative in diesem Entscheidungsproblem heraus.

\begin{table}[h]
	\caption{Nutzwerte der Serverauswahl}
	\begin{center}
		\begin{tabular}{rrrr}
			\toprule
			\multicolumn{1}{l}{ExpressJS} & \multicolumn{1}{l}{SpringBoot} & \multicolumn{1}{l}{Django} & \multicolumn{1}{l}{Flask} \\ \midrule
			                          2,8 &                           1,87 &                       2,66 &                      2,66 \\ \bottomrule
		\end{tabular}
	\end{center}
	\label{tab:server-nutzwerte}
\end{table}


\subsection{Kartenvisualisierung}
\label{sec:kartenvisualisierung}
Die Kriterien zur Auswahl der Kartenvisualisierungs-Bibliothek sind gleich wichtig - daher wird auf die Berechnung des gewichteten Nutzwerts verzichtet.
In \autoref{tab:map-nutzwertanalyse} ist eine kombinierte Übersicht der Bewertung und der Nutzwerte dargestellt.

Leaflet wird bei Kriterium A mit 3 bewertet, da die API sehr einfach zu verwenden war und auch ohne viel Vorwissen kann eine Kartenvisualisierung erstellt werden.
Demgegenüber ist die API von Modestmaps weniger intuitiv und die Dokumentation leicht unübersichtlich.
Polymaps hat eine sehr unintuitive API und erhöht dadurch den Implementierungsaufwand.

Leaflet ist mit 3 zu bewerten, da die Bibliothek stetig weiterentwickelt wird und in dem GitHub Repository regelmäßig Aktivitäten stattfinden \autocite[vgl.][]{leaflet.od}.
Seit 2014 ist auf dem GitHub Repository von Modestmaps keine Aktivität mehr \autocite[vgl.][]{modestmaps.od}.
Seit 2011 ist auf dem GitHub Repository von Polymaps keine Aktivität mehr \autocite[vgl.][]{airship.od} - daher sind Modestmaps und Polymaps in dem Kriterium mit 1 zu bewerten.

Die Punktmarker sind in Leaflet gut anpassbar.
Bei Modestmaps können nur die Grafiken der Punktmarker ausgetauscht werden.
Bei Polymaps gibt es keine Möglichkeit bei Benutzung der Bibliothek die Punktmarker anzupassen, außer die Bibliothek wird angepasst.
Die Aspekte der Anpassbarkeit finden sich in der Bewertung von Kriterium C wieder.

Leaflet ist in dieser Entscheidungssituation die günstigste Alternative.

\begin{table}[h]
	\caption{Nutzwertanalyse zur Kartenvisualisierungs Auswahl}
	\begin{center}
		\begin{tabular}{llrrr}
			\toprule
			Nr.      & Kriterium                      & \multicolumn{1}{l}{Leaflet} & \multicolumn{1}{l}{Modestmaps} & \multicolumn{1}{l}{Polymaps} \\ \midrule
			A        & API                            &                           3 &                              2 &                            1 \\
			B        & Wartung und Weiterentwicklung  &                           3 &                              1 &                            1 \\
			C        & Anpassbarkeit von Punktmarkern &                           3 &                              1 &                            2 \\ \midrule
			Nutzwert &                                &                           9 &                              4 &                            4 \\ \bottomrule
			         &                                &                             &
		\end{tabular}
	\end{center}
	\label{tab:map-nutzwertanalyse}
\end{table}


\subsection{Tile-Service}
\label{sec:tileservice}
Die Kriterien zur Auswahl des Tile-Service haben die gleiche Gewichtung, daher wird auf den gewichteten Nutzwert verzichtet.
In \autoref{tab:tile-nutzerwertanalyse} ist eine kombinierte Übersicht aus Kriterien, Bewertungen und Nutzwert der Alternativen.

Der \ac{OSM} Tile-Service hat die höchste Qualität, da das Kartenmaterial eine detaillierte Straßenkarte ist - daher ist \ac{OSM} bei Kriterum A mit 3 zu bewerten.
Wikimedia Maps bietet auch eine Straßenkarte, aber nicht mit einem so hohen Detailgrad - daher eine Bewertung von 2.
Das \ac{LGL BW} stellt Luftaufnahmen Bereit, welche in einem Tile-Service verwendet werden können.
Die Luftaufnahmen wurden zwischen 2011 und 2017 aufgenommen \autocite[vgl.][]{lglbw.odb}.
Es kann so nicht der aktuelle Zustand der Streuobstwiesen dargestellt werden - daher eine Bewertung von 2.
Géoportail France und Maptiler stellen jeweils einen Satellitenbild Tile-Service bereit. Die Auflösung der Bilder ist nicht verwendbar, da bspw. Bäume nicht zu erkennen sind - daher eine Bewertung von 2. 
Beispielbilder der Entscheidungsalternativen sind in Anhang \ref{sec:karten}.

\ac{OSM}, Wikimedia Maps, Géoportail France und Maptiler stellen einen Tile Service zur direkten Verwendung bereit. Deshalb ist Kriterium B für sie mit 3 zu bewerten \autocite[vgl.][]{ign.od} \autocite[vgl.][]{wikimedia.oda} \autocite[vgl.][]{OMS.oda}.
Bei \ac{LGL BW} können lediglich die Bilder erworben werden. Das bedeutet, dass eigene Infrastruktur aufgebracht werden muss um einen Tile-Service mit den \ac{LGL BW} Daten bereitzustellen \autocite[vgl.][]{lglbw.odb} - daher eine Bewertung mit 1.

\ac{OSM}, Wikimedia Maps und Géoportail France bieten kostenlose Tile-Services an \autocite[vgl.][]{ign.od} \autocite[vgl.][]{wikimedia.oda} \autocite[vgl.][]{OMS.oda} - daher eine Bewertung von 3 bei Kriterium C.
Maptiler und \ac{LGL BW} sind kostenpflichtig \autocite[vgl.][]{lglbw.odb} \autocite[vgl][]{MapTiler.odb} - daher eine Bewertung von 1.

Der Tile-Service von \ac{OSM} ist in dieser Entscheidungssituation die günstigste Entscheidungsalternative.

\begin{table}[h]
	\caption{Nutzwertanalyse zur Tile-Service Auswahl}
	\begin{center}
		\begin{tabular}{llrrrrr}
			\toprule
			Nr.      & Kriterium & \multicolumn{1}{l}{OSM} & \multicolumn{1}{l}{Wikimedia} & \multicolumn{1}{l}{Géoportail France} & \multicolumn{1}{l}{Maptiler} & \multicolumn{1}{l}{LGL BW} \\ \midrule
			A        &  Qualität &                       3 &                             2 &                                     1 &                            1 & 2                          \\
			B        &   Aufwand &                       3 &                             3 &                                     3 &                            3 & 1                          \\
			C        &    Kosten &                       3 &                             3 &                                     3 &                            1 & 1                          \\ \midrule
			Nutzwert &           &                       9 &                             8 &                                     7 &                            5 & 4                          \\ \bottomrule
		\end{tabular}
	\end{center}
	\label{tab:tile-nutzerwertanalyse}
\end{table}

\subsection{Frontend}
In \autoref{tab:frontend-kriterien} ist eine Übersicht der Kriterien zur Bewertung der Entscheidungsalternativen.
Kriterium A und B werden anhand der Vorkenntnisse bewertet.
Die weiteren Kriterien werden mittels der Erfahrung durch Mitarbeit bei der Implementierung der Prototypen bewertet.

\begin{table}[h]
	\caption{Kriterien der Frontend Framework Auswahl}
	\begin{center}
		\begin{tabular}{ll}
			\toprule
			Nr. & Kriterium                                        \\ \midrule
			A   & Vorkenntnisse Programmiersprache                 \\
			B   & Vorkenntnisse Framework                          \\
			C   & Implementierung – Struktur / Nachvollziehbarkeit \\
			D   & Implementierung – Aufwand                        \\
			E   & Implementierung – Benutzeroberfläche             \\
			F   & Dokumentation                                    \\
			G   & Leaflet Integration                              \\ \bottomrule
		\end{tabular}
	\end{center}
	\label{tab:frontend-kriterien}
\end{table}

In \autoref{tab:frontend-gewichtung} ist die Gewichtung der Kriterien beschrieben.
Besondere Wichtigkeit bei der Entscheidung haben die Vorkenntnisse des Entwicklungs-Teams.
Diese sind notwendig, um Ergebnisse für Komponenten zu erzielen, mit denen der Kunde direkt interagiert.

\begin{table}[h]
	\caption{Gewichtung der Kriterien der Frontend Framework Auswahl}
	\begin{center}
		\begin{tabular}{llllllllrr}
			\toprule
			      & A & B & C & D & E & F & G & \multicolumn{1}{l}{Summe} & \multicolumn{1}{l}{Anteil} \\ \midrule
			A     & - & A & A & A & E & A & A &                         5 &                   24,00 \% \\
			B     & - & - & B & B & E & B & B &                         4 &                   19,00 \% \\
			C     & - & - & - & C & C & F & C &                         3 &                   14,00 \% \\
			D     & - & - & - & - & D & F & D &                         2 &                   10,00 \% \\
			E     & - & - & - & - & - & F & E &                         3 &                   14,00 \% \\
			F     & - & - & - & - & - & - & G &                         3 &                   14,00 \% \\
			G     & - & - & - & - & - & - & - &                         1 &                    5,00 \% \\
			Summe &   &   &   &   &   &   &   &                        21 &                  100,00 \% \\ \bottomrule
		\end{tabular}
	\end{center}
	\label{tab:frontend-gewichtung}
\end{table}

In \autoref{tab:frontend-bewertung} ist eine Darstellung der Bewertungen der Alternativen.
Angular bietet durch seine Komponenten-Arten und die klare Umsetzung des Model-View-Controller Konzepts eine gute Struktur und Nachvollziehbarkeit - daher eine Bewertung von 3 bei Kriterium C.
OpenUI5 verliert durch das Component-Konzept und die Manifest Datei an Übersichtlichkeit, dennoch wird auch bei OpenUI5 das Model-View-Controller Konzept klar angewendet - daher eine Bewertung von 2.

Sowohl Angular als auch OpenUI5 erfordern einen beachtlichen Aufwand zur Implementierung der Anforderungen der Prototypen-User-Story. Dies kann sich verstärken, wenn die Benutzeroberflächen ausgereift sein sollen - daher eine Bewertung von 2 bei Kriterium D.

OpenUI5 stellt eine Vielzahl von vorgefertigten Benutzeroberflächen-Komponenten bereit, die ohne Anpassung direkt genutzt werden können - daher eine Bewertung von 3 bei Kriterium E.
Angular stellte mit Angular-Material auch Benutzeroberflächen-Komponenten bereit, aber die Anzahl ist begrenzt und Anpassungen sind zu Nutzung notwendig - daher eine Bewertung von 2 bei Kriterium E.

Die Dokumentation beider Frameworks ist gut und unterstützt bei der Implementierung - daher eine Bewertung von 3 sowohl für Angular, als auch für OpenUI5.

Die Integration von Leaflet ist bei OpenUI5 nicht gegeben - daher eine Bewertung von 1 bei Kriterium G.
Für Angular existiert ein Modul welches Leaflet in Angular verfügbar macht, dennoch ist viel Code notwendig, um dieses Modul zu nutzen - daher eine Bewertung von 2.


\begin{table}[h]
	\caption{Bewertung der Kriterien der Frontend Framework Auswahl}
	\begin{center}
		\begin{tabular}{lrrr}
			\toprule
			Kriterium & \multicolumn{1}{l}{OpenUI5} & \multicolumn{1}{l}{Angular} & \multicolumn{1}{l}{Angular (JavaScript)} \\ \midrule
			A         &                           3 &                           2 &                               3 \\
			B         &                           2 &                           3 &                               3 \\
			C         &                           2 &                           3 &                               3 \\
			D         &                           2 &                           2 &                               2 \\
			E         &                           3 &                           2 &                               2 \\
			F         &                           3 &                           3 &                               3 \\
			G         &                           1 &                           2 &                               2 \\ \bottomrule
		\end{tabular}
	\end{center}
	\label{tab:frontend-bewertung}
\end{table}

In \autoref{tab:frontend-nutzwerte} ist eine Darstellung der Nutzwerte der Handlungsalternativen.
Da Angular und OpenUI5 den gleichen Nutzwert haben, gibt es eine dritte Alternative - Angular (JavaScript).
Bei dieser Alternative gilt die Prämisse, das JavaScript valides TypeScript ist und somit mit JavaScript Kenntnissen auch TypeScript programmiert werden kann.
Dies ändert nur bei Kriterium A den Wert von 2 auf 3.
Somit hat Angular unter dieser Annahme den höchsten Nutzwert und ist somit die günstigste Entscheidungsalternative für dieses Entscheidungsproblem.

\begin{table}[h]
	\caption{Nutzwerte der Frontend Framework Auswahl}
	\begin{center}
		\begin{tabular}{rrr}
			\toprule
			OpenUI5 & Angular & Angular (JavaScript) \\ \midrule
			   2,47 &    2,47 &                 2,71 \\ \bottomrule
		\end{tabular}
	\end{center}
	\label{tab:frontend-nutzwerte}
\end{table}